\documentclass[]{article}
\usepackage{lmodern}
\usepackage{amssymb,amsmath}
\usepackage{ifxetex,ifluatex}
\usepackage{fixltx2e} % provides \textsubscript
\ifnum 0\ifxetex 1\fi\ifluatex 1\fi=0 % if pdftex
  \usepackage[T1]{fontenc}
  \usepackage[utf8]{inputenc}
\else % if luatex or xelatex
  \ifxetex
    \usepackage{mathspec}
  \else
    \usepackage{fontspec}
  \fi
  \defaultfontfeatures{Ligatures=TeX,Scale=MatchLowercase}
\fi
% use upquote if available, for straight quotes in verbatim environments
\IfFileExists{upquote.sty}{\usepackage{upquote}}{}
% use microtype if available
\IfFileExists{microtype.sty}{%
\usepackage{microtype}
\UseMicrotypeSet[protrusion]{basicmath} % disable protrusion for tt fonts
}{}
\usepackage[margin=1in]{geometry}
\usepackage{hyperref}
\hypersetup{unicode=true,
            pdftitle={EX 03 - Data Handling \& MLE},
            pdfauthor={Afek Adler},
            pdfborder={0 0 0},
            breaklinks=true}
\urlstyle{same}  % don't use monospace font for urls
\usepackage{color}
\usepackage{fancyvrb}
\newcommand{\VerbBar}{|}
\newcommand{\VERB}{\Verb[commandchars=\\\{\}]}
\DefineVerbatimEnvironment{Highlighting}{Verbatim}{commandchars=\\\{\}}
% Add ',fontsize=\small' for more characters per line
\usepackage{framed}
\definecolor{shadecolor}{RGB}{248,248,248}
\newenvironment{Shaded}{\begin{snugshade}}{\end{snugshade}}
\newcommand{\AlertTok}[1]{\textcolor[rgb]{0.94,0.16,0.16}{#1}}
\newcommand{\AnnotationTok}[1]{\textcolor[rgb]{0.56,0.35,0.01}{\textbf{\textit{#1}}}}
\newcommand{\AttributeTok}[1]{\textcolor[rgb]{0.77,0.63,0.00}{#1}}
\newcommand{\BaseNTok}[1]{\textcolor[rgb]{0.00,0.00,0.81}{#1}}
\newcommand{\BuiltInTok}[1]{#1}
\newcommand{\CharTok}[1]{\textcolor[rgb]{0.31,0.60,0.02}{#1}}
\newcommand{\CommentTok}[1]{\textcolor[rgb]{0.56,0.35,0.01}{\textit{#1}}}
\newcommand{\CommentVarTok}[1]{\textcolor[rgb]{0.56,0.35,0.01}{\textbf{\textit{#1}}}}
\newcommand{\ConstantTok}[1]{\textcolor[rgb]{0.00,0.00,0.00}{#1}}
\newcommand{\ControlFlowTok}[1]{\textcolor[rgb]{0.13,0.29,0.53}{\textbf{#1}}}
\newcommand{\DataTypeTok}[1]{\textcolor[rgb]{0.13,0.29,0.53}{#1}}
\newcommand{\DecValTok}[1]{\textcolor[rgb]{0.00,0.00,0.81}{#1}}
\newcommand{\DocumentationTok}[1]{\textcolor[rgb]{0.56,0.35,0.01}{\textbf{\textit{#1}}}}
\newcommand{\ErrorTok}[1]{\textcolor[rgb]{0.64,0.00,0.00}{\textbf{#1}}}
\newcommand{\ExtensionTok}[1]{#1}
\newcommand{\FloatTok}[1]{\textcolor[rgb]{0.00,0.00,0.81}{#1}}
\newcommand{\FunctionTok}[1]{\textcolor[rgb]{0.00,0.00,0.00}{#1}}
\newcommand{\ImportTok}[1]{#1}
\newcommand{\InformationTok}[1]{\textcolor[rgb]{0.56,0.35,0.01}{\textbf{\textit{#1}}}}
\newcommand{\KeywordTok}[1]{\textcolor[rgb]{0.13,0.29,0.53}{\textbf{#1}}}
\newcommand{\NormalTok}[1]{#1}
\newcommand{\OperatorTok}[1]{\textcolor[rgb]{0.81,0.36,0.00}{\textbf{#1}}}
\newcommand{\OtherTok}[1]{\textcolor[rgb]{0.56,0.35,0.01}{#1}}
\newcommand{\PreprocessorTok}[1]{\textcolor[rgb]{0.56,0.35,0.01}{\textit{#1}}}
\newcommand{\RegionMarkerTok}[1]{#1}
\newcommand{\SpecialCharTok}[1]{\textcolor[rgb]{0.00,0.00,0.00}{#1}}
\newcommand{\SpecialStringTok}[1]{\textcolor[rgb]{0.31,0.60,0.02}{#1}}
\newcommand{\StringTok}[1]{\textcolor[rgb]{0.31,0.60,0.02}{#1}}
\newcommand{\VariableTok}[1]{\textcolor[rgb]{0.00,0.00,0.00}{#1}}
\newcommand{\VerbatimStringTok}[1]{\textcolor[rgb]{0.31,0.60,0.02}{#1}}
\newcommand{\WarningTok}[1]{\textcolor[rgb]{0.56,0.35,0.01}{\textbf{\textit{#1}}}}
\usepackage{graphicx,grffile}
\makeatletter
\def\maxwidth{\ifdim\Gin@nat@width>\linewidth\linewidth\else\Gin@nat@width\fi}
\def\maxheight{\ifdim\Gin@nat@height>\textheight\textheight\else\Gin@nat@height\fi}
\makeatother
% Scale images if necessary, so that they will not overflow the page
% margins by default, and it is still possible to overwrite the defaults
% using explicit options in \includegraphics[width, height, ...]{}
\setkeys{Gin}{width=\maxwidth,height=\maxheight,keepaspectratio}
\IfFileExists{parskip.sty}{%
\usepackage{parskip}
}{% else
\setlength{\parindent}{0pt}
\setlength{\parskip}{6pt plus 2pt minus 1pt}
}
\setlength{\emergencystretch}{3em}  % prevent overfull lines
\providecommand{\tightlist}{%
  \setlength{\itemsep}{0pt}\setlength{\parskip}{0pt}}
\setcounter{secnumdepth}{0}
% Redefines (sub)paragraphs to behave more like sections
\ifx\paragraph\undefined\else
\let\oldparagraph\paragraph
\renewcommand{\paragraph}[1]{\oldparagraph{#1}\mbox{}}
\fi
\ifx\subparagraph\undefined\else
\let\oldsubparagraph\subparagraph
\renewcommand{\subparagraph}[1]{\oldsubparagraph{#1}\mbox{}}
\fi

%%% Use protect on footnotes to avoid problems with footnotes in titles
\let\rmarkdownfootnote\footnote%
\def\footnote{\protect\rmarkdownfootnote}

%%% Change title format to be more compact
\usepackage{titling}

% Create subtitle command for use in maketitle
\providecommand{\subtitle}[1]{
  \posttitle{
    \begin{center}\large#1\end{center}
    }
}

\setlength{\droptitle}{-2em}

  \title{EX 03 - Data Handling \& MLE}
    \pretitle{\vspace{\droptitle}\centering\huge}
  \posttitle{\par}
    \author{Afek Adler}
    \preauthor{\centering\large\emph}
  \postauthor{\par}
      \predate{\centering\large\emph}
  \postdate{\par}
    \date{2020-03-22}


\begin{document}
\maketitle

Last excercise we did:

\begin{itemize}
\tightlist
\item
  Expectency and Variance of the sample mean and sample sum
\item
  Central limit theoram
\item
  Bias variance decomposition of a point estimator
\item
  Derived an unbiased estimate for \(\sigma^{2}(S^{2})\)
\item
  Covered the student's t-distribution and chi square distribution
\end{itemize}

Today we will:

\begin{itemize}
\tightlist
\item
  Cover methods for point estimattion
\item
  Get to know \texttt{dplyr} package
\item
  Try to develop a feeling for bayesian estimation.
\end{itemize}

\hypertarget{loss-function}{%
\section{Loss function}\label{loss-function}}

A quick recap of the MSE of an estimator:
\[\operatorname{MSE}(\hat{\Theta})=E((\hat{\Theta}-\theta)^{2})\]

The squared loss did not come from heaven but from convienince. for
example, another good criterion can be:

\[\operatorname{MAE}(\hat{\Theta})=E(|\hat{\Theta}-\theta)|)\]

Or many other types of error function. Also, at the lecture you have
seen an \emph{example} of Bayesian estimation where
\(\hat{\theta}_{\mathrm{MMSE}}=\int \theta \mathrm{p}(\theta | \mathbf{x}) \mathrm{d} \theta=\mathrm{E}(\theta | \mathbf{x})\)
,the derivation of this formula was taken under assumption of a square
loss but there are also many other bayesian estimators like the maximum
a posteriori estimation -
\(\hat{\theta}_{\mathrm{MAP}}=\underset{\theta}{\arg \max } p(\mathbf{\theta}| x)\).
In the end of the excercise we will go deeper into this subject.

\hypertarget{point-estimaion}{%
\section{Point estimaion}\label{point-estimaion}}

\hypertarget{some-nice-to-have-charactraists}{%
\subsection{Some nice to have
charactraists}\label{some-nice-to-have-charactraists}}

\begin{itemize}
\tightlist
\item
  Unbiased. If \(E(\hat{\theta}) = \theta\)
\item
  Consistent. If the varaince of the estimator \textasciitilde{}0 when N
  tends to \(\infty\)
\end{itemize}

Remember that the sample mean is unbiased estimator of the population
mean

\hypertarget{point-estimates-with-the-method-of-moments-mom}{%
\subsection{Point estimates with the method of moments
(MOM)}\label{point-estimates-with-the-method-of-moments-mom}}

The first moment
\[E(X)=\frac{\sum_{i=1}^{n} X_{i}}{n} \Rightarrow E(X)=\bar{X}\] The
Second moment
\[E\left(X^{2}\right)=\frac{\sum_{i=1}^{n} X_{i}^{2}}{n} \Rightarrow \]
\[V(X)=E\left(X^{2}\right)-E^{2}(X) \Rightarrow V(X)=\frac{\sum_{i=1}^{n} X_{i}^{2}}{n}-(\bar{X})^{2}\]

\textbf{Q1: MOM}

Let \[X =  \mathcal{U}\left(\theta , \theta + 6 \right)\] Estimate
\(\theta\) with the method of moments\\

Therfore \[E(X) =  (\theta + \theta+ 6 )/2 = \theta +3\] And
\[E(X) = \bar{X} = \theta +3\] By the equation of the first moment.
Therfore \[\hat{\theta} = \bar{X}  -3\]

\hypertarget{point-estimates-with-the-mazimum-likelihood-estimation-mle}{%
\subsection{Point estimates with the mazimum likelihood estimation
(MLE)}\label{point-estimates-with-the-mazimum-likelihood-estimation-mle}}

In statistics, maximum likelihood estimation (MLE) is a method of
estimating the parameters of a probability distribution by maximizing a
likelihood function, so that under the assumed statistical model the
observed data is most probable. The point in the parameter space that
maximizes the likelihood function is called the maximum likelihood
estimate. The logic of maximum likelihood is both intuitive and
flexible, and as such the method has become a dominant means of
statistical inference.

If the likelihood function is differentiable, the derivative test for
determining maxima can be applied. In some cases, the first-order
conditions of the likelihood function can be solved explicitly; for
instance, the ordinary least squares estimator maximizes the likelihood
of the linear regression model. Under most circumstances, however,
numerical methods will be necessary to find the maximum of the
likelihood function. (``Wikipedia'')

\textbf{Q2: MLE} With the binomial distribution - suppose we had a trial
with 49 success ou of 80.

\begin{equation}
L(p)=f_{D}(\mathrm{H}=49 | p)=\left(\begin{array}{c}{80} \\ {49}\end{array}\right) p^{49}(1-p)^{31}\end{equation}
\begin{equation}
0=\frac{\partial}{\partial p}\left(\left(\begin{array}{c}{80} \\ {49}\end{array}\right) p^{49}(1-p)^{31}\right) , \{discard binomial coefficient\}
\end{equation} \begin{equation}
0=49 p^{48}(1-p)^{31}-31 p^{49}(1-p)^{30}, \{(uv)` = u`v +v`u\}
\end{equation} \begin{equation}
=p^{48}(1-p)^{30}[49(1-p)-31 p]
\end{equation} \begin{equation}
=p^{48}(1-p)^{30}[49-80 p]
\end{equation} Can be solved also by applying log on the likelihood.

It's clear that the maximum is at p = 49/80. But let's see how we do it
in R using the bulit in
\href{ehttps://stat.ethz.ch/R-manual/R-devel/library/stats/html/optimize.html}{optimize}
function:

\begin{Shaded}
\begin{Highlighting}[]
\NormalTok{likelihood <-}\StringTok{ }\ControlFlowTok{function}\NormalTok{(p) \{}
\NormalTok{  p}\OperatorTok{^}\DecValTok{49}\OperatorTok{*}\NormalTok{((}\DecValTok{1}\OperatorTok{-}\NormalTok{p)}\OperatorTok{^}\DecValTok{31}\NormalTok{)}
\NormalTok{\}}
\NormalTok{tolerance <-}\StringTok{ }\DecValTok{10}\OperatorTok{^}\NormalTok{(}\OperatorTok{-}\DecValTok{4}\NormalTok{) }
\NormalTok{pmax <-}\StringTok{ }\KeywordTok{optimize}\NormalTok{(likelihood, }\KeywordTok{c}\NormalTok{(}\DecValTok{0}\NormalTok{, }\DecValTok{1}\NormalTok{), }\DataTypeTok{tol =}\NormalTok{ tolerance  , }\DataTypeTok{maximum =}\NormalTok{ T)[[}\DecValTok{1}\NormalTok{]]}
\NormalTok{delta <-}\StringTok{ }\KeywordTok{abs}\NormalTok{(pmax}\OperatorTok{-}\StringTok{ }\NormalTok{(}\DecValTok{49}\OperatorTok{/}\DecValTok{80}\NormalTok{))}
\NormalTok{delta}
\end{Highlighting}
\end{Shaded}

\begin{verbatim}
## [1] 6.814623e-07
\end{verbatim}

\hypertarget{hw1-q3}{%
\section{HW1 q3}\label{hw1-q3}}

\hypertarget{best-pracitces-for-data-data-handling-with-r}{%
\section{Best Pracitces for data Data handling with
R}\label{best-pracitces-for-data-data-handling-with-r}}

R main datatypes:

\begin{itemize}
\tightlist
\item
  vectors
\item
  matrices
\item
  data.frame - matrices with meatadata, added functionallity and allow
  multiple data types
\item
  tibbles - modern take on dataframes
\end{itemize}

\texttt{dplyr} is a grammar of data manipulation, providing a consistent
set of verbs that help you solve the most common data manipulation
challenges:

\begin{itemize}
\tightlist
\item
  \texttt{mutate()} adds new variables that are functions of existing
  variables.
\item
  \texttt{select()} picks variables based on their names.
\item
  \texttt{filter()} picks cases based on their values.
\item
  \texttt{summarize()} reduces multiple values down to a single summary.
\item
  \texttt{arrange()} sorts the rows.
\end{itemize}

\begin{Shaded}
\begin{Highlighting}[]
\KeywordTok{library}\NormalTok{(tidyverse)}
\KeywordTok{library}\NormalTok{(nycflights13)}
\end{Highlighting}
\end{Shaded}

This dataset has 19 columns so the head function is not that usefull
when knitting to html. It is always useful to know how many missing
values we have in our dataset, sometimes missing values are not just
given to us as NA.

\begin{Shaded}
\begin{Highlighting}[]
\KeywordTok{head}\NormalTok{(flights,}\DecValTok{2}\NormalTok{)}
\end{Highlighting}
\end{Shaded}

\begin{verbatim}
## # A tibble: 2 x 19
##    year month   day dep_time sched_dep_time dep_delay arr_time
##   <int> <int> <int>    <int>          <int>     <dbl>    <int>
## 1  2013     1     1      517            515         2      830
## 2  2013     1     1      533            529         4      850
## # ... with 12 more variables: sched_arr_time <int>, arr_delay <dbl>,
## #   carrier <chr>, flight <int>, tailnum <chr>, origin <chr>, dest <chr>,
## #   air_time <dbl>, distance <dbl>, hour <dbl>, minute <dbl>,
## #   time_hour <dttm>
\end{verbatim}

\begin{Shaded}
\begin{Highlighting}[]
\KeywordTok{colSums}\NormalTok{(}\KeywordTok{is.na}\NormalTok{(flights))}\OperatorTok{/}\KeywordTok{nrow}\NormalTok{(flights) }
\end{Highlighting}
\end{Shaded}

\begin{verbatim}
##           year          month            day       dep_time sched_dep_time 
##    0.000000000    0.000000000    0.000000000    0.024511842    0.000000000 
##      dep_delay       arr_time sched_arr_time      arr_delay        carrier 
##    0.024511842    0.025871796    0.000000000    0.028000808    0.000000000 
##         flight        tailnum         origin           dest       air_time 
##    0.000000000    0.007458964    0.000000000    0.000000000    0.028000808 
##       distance           hour         minute      time_hour 
##    0.000000000    0.000000000    0.000000000    0.000000000
\end{verbatim}

\begin{Shaded}
\begin{Highlighting}[]
\KeywordTok{sapply}\NormalTok{(flights,class)}
\end{Highlighting}
\end{Shaded}

\begin{verbatim}
## $year
## [1] "integer"
## 
## $month
## [1] "integer"
## 
## $day
## [1] "integer"
## 
## $dep_time
## [1] "integer"
## 
## $sched_dep_time
## [1] "integer"
## 
## $dep_delay
## [1] "numeric"
## 
## $arr_time
## [1] "integer"
## 
## $sched_arr_time
## [1] "integer"
## 
## $arr_delay
## [1] "numeric"
## 
## $carrier
## [1] "character"
## 
## $flight
## [1] "integer"
## 
## $tailnum
## [1] "character"
## 
## $origin
## [1] "character"
## 
## $dest
## [1] "character"
## 
## $air_time
## [1] "numeric"
## 
## $distance
## [1] "numeric"
## 
## $hour
## [1] "numeric"
## 
## $minute
## [1] "numeric"
## 
## $time_hour
## [1] "POSIXct" "POSIXt"
\end{verbatim}

\begin{center}\rule{0.5\linewidth}{\linethickness}\end{center}

\textbf{At home - find a better way to print the classes and the \% of
missing values in R}

\hypertarget{select-picks-variables-based-on-their-names.}{%
\subsection{\texorpdfstring{\texttt{select()} picks variables based on
their
names.}{select() picks variables based on their names.}}\label{select-picks-variables-based-on-their-names.}}

\begin{Shaded}
\begin{Highlighting}[]
\NormalTok{flight_ditance_airtime <-}\StringTok{ }\NormalTok{flights }\OperatorTok\StringTok{ }\KeywordTok{select}\NormalTok{( distance, air_time) }
\NormalTok{flight_ditance_airtime }\OperatorTok\StringTok{ }\KeywordTok{head}\NormalTok{(}\DecValTok{2}\NormalTok{)}
\end{Highlighting}
\end{Shaded}

\begin{verbatim}
## # A tibble: 2 x 2
##   distance air_time
##      <dbl>    <dbl>
## 1     1400      227
## 2     1416      227
\end{verbatim}

\hypertarget{mutate-adds-new-variables-that-are-functions-of-existing-variables.}{%
\subsection{\texorpdfstring{\texttt{mutate()} adds new variables that
are functions of existing
variables.}{mutate() adds new variables that are functions of existing variables.}}\label{mutate-adds-new-variables-that-are-functions-of-existing-variables.}}

\begin{Shaded}
\begin{Highlighting}[]
\NormalTok{flight_ditance_airtime }\OperatorTok\StringTok{ }\KeywordTok{mutate}\NormalTok{(}\DataTypeTok{mean_speed =}\NormalTok{ distance}\OperatorTok{/}\NormalTok{air_time) }\OperatorTok\StringTok{ }\KeywordTok{head}\NormalTok{(}\DecValTok{2}\NormalTok{)}
\end{Highlighting}
\end{Shaded}

\begin{verbatim}
## # A tibble: 2 x 3
##   distance air_time mean_speed
##      <dbl>    <dbl>      <dbl>
## 1     1400      227       6.17
## 2     1416      227       6.24
\end{verbatim}

If you only want to keep the new variables, use \texttt{transmute()}:

\begin{Shaded}
\begin{Highlighting}[]
\NormalTok{flight_ditance_airtime }\OperatorTok\StringTok{ }\KeywordTok{transmute}\NormalTok{(}\DataTypeTok{mean_speed =}\NormalTok{ distance}\OperatorTok{/}\NormalTok{air_time) }\OperatorTok\StringTok{ }\KeywordTok{head}\NormalTok{(}\DecValTok{2}\NormalTok{)}
\end{Highlighting}
\end{Shaded}

\begin{verbatim}
## # A tibble: 2 x 1
##   mean_speed
##        <dbl>
## 1       6.17
## 2       6.24
\end{verbatim}

\hypertarget{filter-picks-cases-based-on-their-values.}{%
\subsection{\texorpdfstring{\texttt{filter()} picks cases based on their
values.}{filter() picks cases based on their values.}}\label{filter-picks-cases-based-on-their-values.}}

\begin{Shaded}
\begin{Highlighting}[]
\NormalTok{flights }\OperatorTok\StringTok{ }\KeywordTok{filter}\NormalTok{(}\KeywordTok{is.na}\NormalTok{(dep_delay)) }\OperatorTok\StringTok{ }\KeywordTok{head}\NormalTok{(}\DecValTok{2}\NormalTok{)}
\end{Highlighting}
\end{Shaded}

\begin{verbatim}
## # A tibble: 2 x 19
##    year month   day dep_time sched_dep_time dep_delay arr_time
##   <int> <int> <int>    <int>          <int>     <dbl>    <int>
## 1  2013     1     1       NA           1630        NA       NA
## 2  2013     1     1       NA           1935        NA       NA
## # ... with 12 more variables: sched_arr_time <int>, arr_delay <dbl>,
## #   carrier <chr>, flight <int>, tailnum <chr>, origin <chr>, dest <chr>,
## #   air_time <dbl>, distance <dbl>, hour <dbl>, minute <dbl>,
## #   time_hour <dttm>
\end{verbatim}

\hypertarget{arrange-picks-cases-based-on-their-values.}{%
\subsection{\texorpdfstring{\texttt{arrange()} picks cases based on
their
values.}{arrange() picks cases based on their values.}}\label{arrange-picks-cases-based-on-their-values.}}

\begin{Shaded}
\begin{Highlighting}[]
\NormalTok{flights }\OperatorTok\StringTok{ }\KeywordTok{arrange}\NormalTok{(}\KeywordTok{desc}\NormalTok{(month)) }\OperatorTok\StringTok{ }\KeywordTok{head}\NormalTok{(}\DecValTok{2}\NormalTok{)}
\end{Highlighting}
\end{Shaded}

\begin{verbatim}
## # A tibble: 2 x 19
##    year month   day dep_time sched_dep_time dep_delay arr_time
##   <int> <int> <int>    <int>          <int>     <dbl>    <int>
## 1  2013    12     1       13           2359        14      446
## 2  2013    12     1       17           2359        18      443
## # ... with 12 more variables: sched_arr_time <int>, arr_delay <dbl>,
## #   carrier <chr>, flight <int>, tailnum <chr>, origin <chr>, dest <chr>,
## #   air_time <dbl>, distance <dbl>, hour <dbl>, minute <dbl>,
## #   time_hour <dttm>
\end{verbatim}

\hypertarget{summarize-reduces-multiple-values-down-to-a-single-summary.}{%
\subsection{\texorpdfstring{\texttt{summarize()} reduces multiple values
down to a single
summary.}{summarize() reduces multiple values down to a single summary.}}\label{summarize-reduces-multiple-values-down-to-a-single-summary.}}

\begin{Shaded}
\begin{Highlighting}[]
\NormalTok{by_month <-}\StringTok{ }\KeywordTok{group_by}\NormalTok{(flights,month)}
\NormalTok{by_month }\OperatorTok\StringTok{ }\KeywordTok{summarise}\NormalTok{(}\DataTypeTok{count =} \KeywordTok{n}\NormalTok{()) }\OperatorTok\StringTok{ }
\StringTok{  }\KeywordTok{ggplot}\NormalTok{( }\DataTypeTok{mapping =} \KeywordTok{aes}\NormalTok{(}\DataTypeTok{x =}\NormalTok{ month, }\DataTypeTok{y =}\NormalTok{ count)) }\OperatorTok{+}\StringTok{ }\KeywordTok{geom_bar}\NormalTok{(}\DataTypeTok{stat=}\StringTok{"identity"}\NormalTok{) }\OperatorTok{+}\StringTok{  }\KeywordTok{coord_cartesian}\NormalTok{(}\DataTypeTok{ylim =} \KeywordTok{c}\NormalTok{(}\DecValTok{2}\OperatorTok{*}\DecValTok{10}\OperatorTok{^}\DecValTok{4}\NormalTok{, }\DecValTok{3}\OperatorTok{*}\DecValTok{10}\OperatorTok{^}\DecValTok{4}\NormalTok{))}
\end{Highlighting}
\end{Shaded}

\includegraphics{03-_Point_estimation_and_dplyr_package_files/figure-latex/summarize method-1.pdf}

\begin{Shaded}
\begin{Highlighting}[]
\KeywordTok{ggplot}\NormalTok{(}\DataTypeTok{data =}\NormalTok{ flights) }\OperatorTok{+}\StringTok{ }
\StringTok{  }\KeywordTok{geom_bar}\NormalTok{(}\DataTypeTok{mapping =} \KeywordTok{aes}\NormalTok{(}\DataTypeTok{x =}\NormalTok{ month)) }\OperatorTok{+}
\StringTok{ }\KeywordTok{coord_cartesian}\NormalTok{(}\DataTypeTok{ylim =} \KeywordTok{c}\NormalTok{(}\DecValTok{2}\OperatorTok{*}\DecValTok{10}\OperatorTok{^}\DecValTok{4}\NormalTok{, }\DecValTok{3}\OperatorTok{*}\DecValTok{10}\OperatorTok{^}\DecValTok{4}\NormalTok{))}
\end{Highlighting}
\end{Shaded}

\includegraphics{03-_Point_estimation_and_dplyr_package_files/figure-latex/another way-1.pdf}

Additional resources:

\begin{itemize}
\tightlist
\item
  \href{https://r4ds.had.co.nz/transform.html}{r4ds}
\item
  \href{https://dplyr.tidyverse.org/}{dplyr}
\item
  \href{https://github.com/rstudio/cheatsheets/blob/master/data-transformation.pdf}{dplyr
  cheat sheet}
\end{itemize}

\hypertarget{bayesian-estimation}{%
\section{Bayesian estimation}\label{bayesian-estimation}}

We want to minimize with respect to a given loss function -
\[\int L(\hat{\theta}- \theta)*p(\theta|x)d\theta\]

In the lecture, we have seen that when
\(L(\hat{\theta},\theta) = (\hat{\theta}- \theta)^2\) than
\(\hat{\theta} = E(posterier)\), other types of loss functions will
derive different estimators (like we have seen above). The logic of this
method is as follows - we have somekind of a distribution over
\(\theta\), but we need to choose only one of those. So we formulate an
objective function and minimze it with respect to the parameter that we
want to find. The most used ones are the mean, median, and common of
that distribution, and as we said, thet are the bayesian estimators for
different loss functions.


\end{document}
